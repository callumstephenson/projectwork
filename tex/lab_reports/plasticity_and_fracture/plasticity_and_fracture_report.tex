\documentclass[12pt]{article}

% import packages
\usepackage[utf8]{inputenc}
\usepackage{float}
\usepackage{subfloat}
\usepackage{subfig}
\usepackage{amsmath}
\usepackage[toc,page]{appendix}
\usepackage{caption}
\usepackage{graphicx}
\usepackage{hyperref}
\usepackage[left=1in,top=0.75in,right=1in,bottom=0.75in]{geometry}

\graphicspath{ {./images} }
% hyperref setup
\hypersetup{
	colorlinks=true,
	linkcolor=blue,
	filecolor=blue,      
	urlcolor=blue,
	citecolor=blue,
	pdftitle={Plasticity and fracture report},
	pdfpagemode=FullScreen,
}
% titlepage	
\title{\textbf{Plasticity and fracture report}}
\author{Callum Stephenson, css47, GP173, Trinity}
\date{}

\begin{document}
    \begin{titlepage}
        \maketitle
        \thispagestyle{empty}
    \end{titlepage}
    \section{Questions \& answers}
        Material tested by our group: copper \\ \\
        \subsection{Calculation}
            (a)(i) $\sigma_y$ was around 3e8 $\frac{N}{m^2}$ \\ \\
            (a)(ii) $\sigma_{ts}$ was around 3.5e8 $\frac{N}{m^2}$ \\ \\
            (a)(iii) Strain to fracture ratio was 22\% \\ \\
            (b) $E = \frac{1\div2e-5}{240e-6} = 208$ GPa \\ \\
            (2)(i) $\sigma_y$ was around 3.75e8 $\frac{N}{m^2}$  \\ \\
            (2){ii} $\sigma_{ts}$ was around 4.5e8 $\frac{N}{m^2}$ \\ \\
            (2)(iii) Strain to fracture was 25\% \\ \\
            (3)(i) $\sigma_y$ was around 3.08e8 $\frac{N}{m^2}$ \\ \\
            (3)(ii)$\sigma_{ts}$ was around 3.59e8 $\frac{N}{m^2}$ \\ \\
            (3)(iii) Strain to fracture was 19\%
        \subsection{Analysis}
            The value of E obtained for steel using the strain gauge was 208 GPa, the error is 0.95\%. The value for E accounting for
            machine characteristic from my data is 1.71e11 (170 GPa), which was an error of 30\% , the total uncertainty usign this
            method is around $\pm$20\% hence it is not the best method for measuring the young's modulus of a sample. \\ \\
            The original volume of the shank was 504mm$^3$, however during the plastic deformation the sample's length increased to 30.7mm, and 
            the crossectional area reduced to 16.4mm in the section which did not experience necking, and hence volume is conserved.  \\ \\
            Increases in yield stress and tensile strength with work hardening - 2.7\% increase in yield stress and 2.6\% increase in yield strength.
            Percentage reduction in s-t-f: 13.6 \% reduction in s-t-f ratio. 
            Metal becomes less ductile when work hardened as it is hard for the dislocations to move over one another when there are already
            multiple jogs, this means it takes more energy to slide over one another. \\ \\
            If the initial sample were twice the length, the extension would be multiplied by 2 as it is measured in mm and not strain. Yield
            stress and ductility would have the same values as it would still take the same amount of force to reach the plastic region, and again
            the same amount of strain to failure ratio(as strain is a relative percentage and not an absolute measurement). \\ \\
            During cooling the yield stress of copper increases 25\%, copper undergoes a ductile failure whereas steel undergoes a brittle failure
            at the low temperature. Copper has a large extension with very little extra force once it reaches its plastic region when cooled to -196,
            but the steel does not.
\end{document}