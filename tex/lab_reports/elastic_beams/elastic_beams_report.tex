\documentclass[12pt]{article}

% import packages
\usepackage[utf8]{inputenc}
\usepackage{float}
\usepackage{subfloat}
\usepackage{subfig}
\usepackage{amsmath}
\usepackage[toc,page]{appendix}
\usepackage{caption}
\usepackage{graphicx}
\usepackage{hyperref}
\usepackage[left=1in,top=0.75in,right=1in,bottom=0.75in]{geometry}

\graphicspath{ {./images} }
% hyperref setup
\hypersetup{
	colorlinks=true,
	linkcolor=blue,
	filecolor=blue,      
	urlcolor=blue,
	citecolor=blue,
	pdftitle={},
	pdfpagemode=FullScreen,
}
% titlepage	
\title{\textbf{}}
\author{Callum Stephenson, css47}
\date{22nd February 2022}

\begin{document}
    \begin{titlepage}
        \maketitle
        \thispagestyle{empty}
        \vspace{13cm}
        \textbf{Department of Engineering, University of Cambridge}
    \end{titlepage}
    \newpage
    \section{Summary}
    \section{Introduction}
        The aim of this report is to investigate the nature of beams bending under a point load within the elastic region. These deflections are so 
        small that it is necessary to set up equipment to measure them precisely. 
        A device will be used to measure the local curvature of the beam rather than placing multiple displacement gauges along the length. \\ \\
        The way in which curvative is investigated is using a simply supported aluminium beam loaded in the midpoint. 
    \section{Laboratory setup}
        \subsection{Apparatus}
            Within this section, the equipment used within the laboratory will be listed with reasoning why it was chosen and a short summary of its functionality. \\
            The beam itself was an aluminium beam with notches every 5 centimetres, with notch 1 and 11 to be placed on a roller and pin joint respectively. This were to be
            placed into a loading frame. \\
            A curvature gauge was used in order to measure the local curvature at each of the 14 points in the beam. The curvature gauge was datumed on a known flat beam
            in order to remove zero error from the experiment. They were already previously calibrated on a large wheel of known curvature close to where they are stored.\\
            In order to measure the central displacement another separate displacement gauge was also required. There was no need to zero this as the measurement
            was simply a difference rather than an absolute with refenced to zero. \\
            A load stalk and 5 kilogram masses were required in order to achieve the desired load on the midpoint of the beam.
        \subsection{Testing methodology}
            Prior to adding any load to the setup, the initial curvature must be measured. The empty load stalk was added and the digital curvature gauge was 
            placed at each of the 14 points along the beam and a measurement was taken and recorded into a table. \\ After recording the pre-load curvature, 5 kilogram masses
            were added to the load stalk. The curvature was once again recorded at each of the 14 measurement points along the beam. Finally, once the midpoint displacement
            was measured with the load added using the analogue displacement gauge, then the setup was unloaded and a second reading was taken. The difference between these
            two measurements was recorded - this is mainly to have a comparison for the calculation later.
    \section{Results and observations}

    \section{Conclusion}
\end{document}